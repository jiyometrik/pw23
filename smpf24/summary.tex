\documentclass[11pt, a4paper]{pancake-article}

\usepackage[margin=1in]{geometry}

\usepackage{mathtools}

\usepackage{booktabs, csvsimple}

\usepackage{siunitx}
\sisetup{mode=math}

\title{Sorting sentiments of hotel reviews through machine learning: a summary}
\author{Yap Hao Ming Darren \and Tan Jia He \and Fu Jinghang}
\date{\scshape Singapore Mathematics Project Festival 2024}

% start the document!
\begin{document}

\pagestyle{plain}

\maketitle

% ** abstract **
\begin{abstract}
	Hotels depend on tourists to survive. They gather consumer opinions via customer
	reviews to improve their services. This project used sentiment analysis---the
	process of determining the emotional tone of text---to quantify consumers'
	opinions on hotels, and predicted the overall sentiment of hotel reviews.
	International hotel reviews in English were split into individual words,
	assigning each word a score based on its relative intensity. The sentiment
	makeup of the review dataset was highlighted and the most frequent tokens were
	identified. Two machine learning models---a logistic regression model and a random
	forest classifier---were also constructed to predict the overall sentiment of a
	review. It was shown that the models were capable of predicting the overall
	sentiment of hotel reviews. This project highlights the possibility of using
	sentiment analysis models to create applications that allow customers to better
	understand the perceived quality of a hotel and potentially combat review fraud---the
	the dishonest practice of manipulating false reviews to boost a hotel's ratings.
\end{abstract}

\section{Introduction}
After the Singapore government relaxed travel restrictions due to COVID-19,
there has been a recent increase in the number of tourists travelling in and out of Singapore.
As such, hotels have seen a rise in the number of prospective tourists to be housed,
and this may encourage an increase in the number of reviews hotels may receive.

Today, it is common to use social networks, messengers, and review websites
to receive data from customer opinions. This is especially true for hotels,
where previous occupants may evaluate the hotel on several factors through their
reviews---be it cleanliness, facilities, location and convenience, etc.
These come in two forms---quantitative reviews (based on stars, diamonds,
hearts, etc.) and qualitative reviews through text.

However, quantitative reviews do not always paint the full picture of customers'
opinions towards a certain hotel. Though it is certainly helpful to have a more
objective rating system using numerical scores, eg. the Department of Tourism
grading system in the Philippines, or the European Hotelstars Union system,
these are given by customers subjectively and do not reflect the reasons for
customers giving the rating. There is also evidence of manipulation of ratings
by hotel management itself, where hotels may be compelled to forge positive or
negative ratings to bias the overall rating.
Therefore, we propose using sentiment analysis to extract customers' true
feedback on hotels instead.

\section{Objectives}
The objectives of this research were as such:
\begin{enumerate}
	\item To run sentiment analysis on individual words and quantify them on a numerical scale
	\item To run sentiment analysis on paragraphs and quantify them on a numerical scale
	\item To use sentiment analysis on hotel reviews to determine consumers' overall opinions of hotels
\end{enumerate}

\section{Methods and Results}

\subsection{Token-based sentiment analysis}\label{sec:tokens}
It was found that the vast majority
of tokens in the dataset held a positive sentiment  after a large number of neutral tokens were disregarded.
The high number of neutral tokens can be attributed to nouns
describing the hotel, like \textit{food}, \textit{hotel} and \textit{pool}.

\subsection{Review-based sentiment analysis}\label{sec:reviews}

The software SentiStrength was used to generate two scores for each review:
It is also observed that a large proportion of reviews were overall positive (\qty{70}{\percent})
and only a handful of negative reviews were present (\qty{13}{\percent}) in the dataset.
Though neutral reviews may seem to make up a sizeable proportion of the reviews, this merely
shows that the magnitude of the positive and negative sentiment scores is equal.

\subsection{Predicting sentiment via token frequency}

For each review, a feature vector was constructed, which contains how
relevant each token in a review is in determining the sentiment of the review
---the term frequency--inverse document frequencies of each token (Equation~\ref{eq:tf-idf}).

\begin{equation}
	\text{tfidf}\left(t, d, D\right) = \text{tf}\left(t, d\right)\log\frac{N}{1 + \left|\left\{d \in D : t\in d\right\}\right|}
	\label{eq:tf-idf}
\end{equation}

where \(t\), \(d\) and \(D\) represent a token, a review and the collection of reviews we have. This function is vital
in normalising textual inputs to numerical values that can be modelled mathematically.

Since positive reviews greatly outnumbered negative ones within the dataset, a logistic regression model was used to
determine optimal class weights for both the \textit{positive} and \textit{negative} classes. Its performance was then evaluated using its $F_1$ score (Equation~\ref{eq:f1}), which measures both the relevance of the prediction models made and its sensitivity to false positive predictions.

\begin{equation}
	F_1 = \frac{2 T_p}{2 T_p + F_p + F_n}
	\label{eq:f1}
\end{equation}

A large variety of machine learning models were constructed
to predict the sentiment polarity of a review (its positive or negative nature) given the aforementioned tf--idf vectors, with
the class weights obtained taken into consideration. Their $F_1$ scores were evaluated.

It was shown that all constructed models performed well in all three metrics, being
able to make true positive predictions much of the time, distinguish true and false
positive predictions with great success and return relevant results. Notably, the KMeans
model performs the worst out of all models: due to the unsupervised nature of the prediction
model, the target class of each review (either negative or positive) was not revealed to it,
leading it to make more incorrect predictions.
\end{document}
