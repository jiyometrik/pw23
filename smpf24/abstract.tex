\documentclass[12pt, a4paper]{pancake-article}

\usepackage{mathtools}

\usepackage{siunitx}
\sisetup{mode=math}

\usepackage{setspace}
\onehalfspacing

% colors!
\usepackage[usenames, dvipsnames]{xcolor}
\newcommand{\tint}[1]{{\color{accent}#1}}

\usepackage{hyperref}
\hypersetup{
	hidelinks,
	colorlinks,
	breaklinks=true,
	urlcolor=accent,
	citecolor=accent,
	linkcolor=accent,
	bookmarksopen=false,
	pdftitle={Sorting sentiments of hotel reviews through machine learning},
	pdfauthor={Darren Yap},
}

\title{Sorting \tint{sentiments} of hotel reviews through \tint{machine learning}}
\author{\tint{Yap} Hao Ming Darren \and \tint{Tan} Jia He \and \tint{Fu} Jinghang}
\date{\scshape Singapore Mathematics Project Festival 2024}

% start the document!
\begin{document}

\maketitle
\begin{abstract}
	Hotels depend on tourists to survive. They gather consumer opinions via customer
	reviews to improve their services. This project used sentiment analysis---the
	process of determining the emotional tone of text---to quantify consumers'
	opinions on hotels, and predicted the overall sentiment of hotel reviews.
	International hotel reviews in English were split into individual words,
	assigning each word a score based on its relative intensity. The sentiment
	makeup of the review dataset was highlighted and the most frequent tokens were
	identified. Two machine learning models---a logistic regression model and a random
	forest classifier---were also constructed to predict the overall sentiment of a
	review. It was shown that the models were capable of predicting the overall
	sentiment of hotel reviews. This project highlights the possibility of using
	sentiment analysis models to create applications that allow customers to better
	understand the perceived quality of a hotel and potentially combat review fraud---the
	the dishonest practice of manipulating false reviews to boost a hotel's ratings.
\end{abstract}

\end{document}
