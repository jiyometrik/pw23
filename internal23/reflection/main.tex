\documentclass[12pt,a4paper,bibliography=totocnumbered]{scrartcl}
\usepackage{xcolor} % add some colour to life!
\usepackage{pagecolor}
\definecolor{background}{HTML}{F8F0E3}
\definecolor{textcolor}{HTML}{4D331B}
\definecolor{texthigh}{HTML}{CBA83C}
\newcommand{\high}[1]{{\color{texthigh} #1}}
\newcommand{\highit}[1]{\it{\high{#1}}}

% headings
\usepackage{scrlayer-scrpage}
\ihead{Group 8--01}
\ohead{Sorting sentiments of hotel reviews through machine learning}

% tables
\usepackage{multicol}
\usepackage{array}

% fonts
\usepackage{microtype} % miniature font tweaks
\usepackage[scale=1.25]{AlegreyaSans}
\usepackage{CharisSIL}

\def\bf#1{\textbf{#1}}
\def\it#1{\textit{#1}}
\usepackage{setspace}

\subject{Project Work---Individual Reflection}
\title{Sorting sentiments of hotel reviews through machine learning}
\subtitle{Group 8--01}
\author{Yap Hao Ming Darren (3i227, L)}
\date{2023}
\pagecolor{background}
\color{textcolor}
\begin{document}
\doublespacing
\maketitle
\tableofcontents
\section{Individual Reflection}
Embarking on a mathematics research project
that delved into the world of sentiment analysis
using machine learning techniques was both exhilarating
and nerve-wracking for me, as someone who hadn't yet ventured
into the realm of large-scale machine learning projects.
I had only a rudimentary understanding of how machine
learning was to be implemented into code, much less
the mathematical theories behind them that tied
this machine learning endeavour with the project category---Category 8.

Our project revolved around using natural language processing
(NLP) techniques to analyse the sentiments of hotel reviews,
ranking them on a tripartite scale of positive, neutral
and negative sentiment. After which, we would use
the tokens in hotel reviews themselves to predict
the overall sentiment of a hotel review, using machine learning
models.

I was relatively unfamiliar with machine learning
techniques upon starting the project, so I searched
seemingly endlessly for similar examples conducted
by tertiary researchers online to get a grasp
of the fundamentals. Fortunately, I also sought
the guidance of my peers who were more adept at
machine learning, and consulted them
on glitches and bugs within
the machine learning algorithms I had implemented.

One of the major issues that we faced in the course
of the project was during its conception. We knew
we were going to focus on data analysis and
statistical probability as our main area of research,
but were not sure on what. After repeated
consultations with our mentor, we decided to conduct
the study on hotel reviews, since the post-COVID
era has brought in a new wave of potential tourists
from abroad. We had chanced upon a large
dataset of ten thousand hotel reviews
from international hotels in English from Kaggle,
making the idea of analysing hotel reviews opportune.

By the end of the project, we had compared two models, the
\it{random forest classifier} and the \it{logistic regression model}.
By exploring the two different ways these models made predictions,
we produced valuable insights regarding the models' precisions
and accuracies.

In conclusion, despite the challenges and frustrations
we faced in our project initially, we pulled through
with insightful findings and a creative application
of Mathematics in machine learning techniques. We
had evaluated the reasons behind such findings carefully
and learnt to exercise critical thinking and creative
skills in this project's development, making
the project even more fulfilling.
\end{document}