\documentclass[12pt,bibliography=totocnumbered]{scrartcl}
\usepackage{pw23}
\usepackage{scrlayer-scrpage}
\ihead{\it{Group 8--01}}
\ohead{\it{Sorting Sentiments}}

\subject{SMTP (Mathematics) Written Report}
\title{Sorting Sentiments}
\subtitle{Group 8--01}
\author{
	Fu Jinghang (3i104) \and
	Tan Jia He (3i325) \and
	Yap Hao Ming Darren (3i227, L)
}
\publishers{Hwa Chong Institution (High School)}
\date{2023}

\begin{document}
\pagestyle{scrheadings}
\doublespacing

% title, toc.
\maketitle
\pagebreak
\tableofcontents
\pagebreak

\section{Introduction and Rationale}
After the Singapore government relaxed travel restrictions due to COVID-19,
there has been a recent increase in the number of tourists travelling in and out of Singapore.
As such, hotels have seen a rise in the number of prospective tourists to be housed,
and this may encourage an increase in the number of reviews hotels may receive.

Today, it is common to use social networks, messengers, and review websites
to receive data from customer opinions. This is especially true for hotels,
where previous occupants may evaluate the hotel on several factors through their
reviews --- be it cleanliness, facilities, location and convenience, etc.
These come in two forms --- a quantitative review (based on stars, diamonds, hearts, etc.)
and a more qualitative review through text.

However, quantitative reviews do not always paint the full picture of customers'
opinions towards a certain hotel. Though it is certainly helpful to have a more
objective rating system using numerical scores, eg. the Department of Tourism grading system in
the Philippines, or the European Hotelstars Union system,
these are given by customers subjectively and do not reflect the reasons for customers giving the rating.
There is also evidence of manipulation of ratings by hotel management itself, where hotels may be
compelled to forge positive or negative ratings to bias the overall rating.
This made up $2.1\%$ of the $66$ million reviews submitted to TripAdvisor (TripAdvisor, 2019). % TODO cite!
Therefore, we propose using sentiment analysis to extract customers' true feedback on hotels instead.

\section{Objectives and Research Questions}
\subsection{Objectives}
\begin{enumerate}
	\item To run sentiment analysis on individual words and quantify them on a numerical scale
	\item To run sentiment analysis on paragraphs and quantify them on a numerical scale
	\item To use sentiment analysis on hotel reviews to determine consumers' overall opinions of hotels
\end{enumerate}

\subsection{Research Questions}
\begin{enumerate}
	\item How could we quantify the sentiments of individual words on a numerical scale?
	\item How could we quantify the sentiments of paragraphs on a numerical scale?
	\item How could we use tokens in hotel reviews to predict the overall sentiment of the review?
\end{enumerate}

\subsection{Fields of Math}
\begin{itemize}
	\item Data Science
	\item Machine Learning
	\item Probability and Statistics
\end{itemize}

\subsection{Terminology}
Below is a listing of the terminology, mostly pertaining to sentiment
analysis, used in this report.

\begin{table}[htpb] \caption{Terminology used in this report.}
	\label{tab:terms}
	\begin{center}
		\begin{tabular}[]{|p{.3\textwidth}|p{.7\textwidth}|}
			\hline
			\bf{Term}                        & \bf{Definition}                                                                                                                                              \\
			\hline
			Sentiment analysis               & Process of determining the emotional tone of text                                                                                                            \\
			\hline
			Token                            & A unit of meaning (usually a word) that carries sentiment                                                                                                    \\
			\hline
			Tokenise                         & To split a piece of text up into its constituent tokens, to be used for further analysis.                                                                    \\
			\hline
			Lemmatise                        & To sort, so as to group together, inflected or variant forms of the same word, eg. `watching', `watchful', and `watched'                                     \\
			\hline
			Stop words                       & Tokens that carry little meaning during sentiment identification, such as `is', `I', `that', etc.; that should be filtered out before the processing of text \\
			\hline
			Lexicon                          & A dictionary that maps singular tokens to a category of sentiments or sentiment scores                                                                       \\
			\hline
			Polarity                         & Whether a piece of text is positive, negative or neutral in sentiment                                                                                        \\
			\hline
			Bipartite sentiment              & Sentiment which is grouped into two categories, usually positive and negative                                                                                \\
			\hline
			Tripartite sentiment             & Sentiment which is grouped into three categories, usually positive, negative and neutral                                                                     \\
			\hline
			Sentiment score                  & A number that shows the overall sentiment of a token or a piece of text                                                                                      \\
			\hline
			Lexicon-based sentiment analysis & A method of sentiment analysis which sorts tokens into categories, aided by a lexicon, then calculates the overall sentiment score of a piece of text        \\
			\hline
			Corpus-based sentiment analysis  & A method of sentiment analysis which relies on the co-occurrences of tokens within a piece of text itself, rather than relying on an external lexicon        \\
			\hline
		\end{tabular}
	\end{center}
\end{table}

\pagebreak

\section{Literature Review}
Sentiment analysis is a field of study which utilises computational methods to analyse text,
and then categorises the text, usually into three main polarities ---  positive, neutral
and negative. It has broad applications which range from determining consumers' opinion in
sales and product analysis, to competitor research in marketing, and even detecting public
opinion in social media monitoring. Sentiment analysis will be used in this project to
analyse hotel occupants' reviews, and also determine the most significant upsides
and downsides of each hotel, without interference from human bias.

% TODO cite
Sohangir et al. (2018) predicted the stock market opinion of StockTwits communities
of expert investors. Sentiment analysis was used in a Deep Learning model to extract
sentiment from Big Data. A Pearson Correlation Coefficient combined the linear correlation
between users' sentiment and future stock prices, which
proved the accuracy of user sentiment to $53\%$.
It was concluded that convolutional neural networks (CNN), a Deep Learning algorithm,
was able to predict stock market movement based on sentiment.

% TODO cite
Using the social networking site Twitter, Villavicencio et al. (2021)
determined Filipinos' sentiment in response to the Philippine government's efforts
at tackling COVID--19, specifically the implementation of vaccination. Natural
Language Processing (NLP) techniques such as sentiment analysis were used to
extract sentiment from text in the English and Filipino languages, which was used to
train a Naïve Bayes model. A confusion matrix was produced, representing the
prediction accuracy of the Naïve Bayes model ($81.77\%$) at classifying
sentiment into positive, neutral and negative categories. It was concluded that
sentiment analysis towards COVID-19 vaccines were very accurate, even helping the
Philippine government better conduct budget planning and coordinate COVID-19 efforts.

% TODO cite
Borrajo-Millán et al. (2021) \cite{spain} analysed tourism quality in Spain by
extracting sentiment from reviews by Chinese people on the tourism
social networking sites Baidu Travel, Ctrip, Mafengwo, and Qunar.
Two sentiment analysis methods, lexicon-matching and corpus-based machine
learning methods, were used. These methods allow the processing of
unstructured text of comparatively longer lengths. Clustered data
visualisation categorised aspects of Spanish tourism into positive
and negative groups, with the majority residing with positive sentiment.
It was concluded that sentiment analysis can be used to improve tourism
quality and sustainability decision-making.

% TODO cite
Guzman et al. (2014b) used SentiStrength, a tool for lexical sentiment
analysis --- sentiment analysis done on short, low quality texts --- to study
emotions expressed in GitHub commit comments of different open-source projects.
Their method involved assigning scores to each word, then calculating the net
score for each comment. SentiStrength splits each comment into snippets, assigns
each a score by computing the maximum and minimum scores of the sentences it contains.
Following which, the average of the positive and negative scores is taken as the
sentiment score of the entire commit. This study showed that Java projects warranted
more negative comments, and projects which had more distributed teams tended
to have a higher positive sentiment.

In conclusion, the literature reviewed showed many possible applications
of sentiment analysis in quantifying the underlying emotion of feedback
on online platforms. Lexicon-based sentiment analysis, which assigns each
word a sentiment, then calculates a sentence's total sentiment score, can
be used, due to its simplicity in implementation, and the availability of
many open-source sentiment lexicons. In addition, sentiment categorisation
using lexicon-based sentiment analysis makes accurate predictions upwards of
$70\%$ of the time (Khoo et al., 2017). SentiStrength would also be useful for % TODO cite!
detecting sentiment from hotel reviews which are usually short in length quickly
and efficiently, optimising the process of extracting sentiment from tourists'
reviews of hotels. Using SentiStrength for sentiment generation is also
rather accurate, generating both positive and negative sentiments with more
than $60\%$ accuracy (Thelwall et al., 2010). Therefore, the strategies listed % TODO cite!
above could be adopted or emulated on a smaller scale for this project.

\section{Methodology}
\subsection{Research Question 1}
\it{How could we quantify the sentiments of individual words on a numerical scale?}
\subsection{Research Question 2}
\it{How could we quantify the sentiments of paragraphs on a numerical scale?}
\subsection{Research Question 3}
\it{How could we use tokens in hotel reviews to predict the overall sentiment of a review?}
\subsubsection{Logistic Regression Model}
A \bf{logistic regression model} was constructed to classify the reviews collected into
two categories: those with a net positive sentiment and those with a net negative sentiment.
The reviews were split into training, testing and prediction sets.

Each review was tokenized and lemmatised. A feature vector for each review was
then constructed, which shows how often each word occurs in the particular review---the
term frequencies for each word, where $\text{tf}(t,d)$ represents the relative term frequency of
token $t$ in review $d$:

\begin{equation}
	\text{tf}(t,d) = \frac{f_{t,d}}{\sum_{t'\in d}^{}f_{t',d}}
	\label{eq:tf}
\end{equation}

where $f_{t,d}$ is the raw count of a token $t$ in review $d$. The denominator of the above
function is simply the total number of tokens in review $d$.

To evaluate the relevance of a token in determining the overall sentiment of the
review, the \bf{term frequency-inverse document frequency} (\it{tf-idf}) is found, where:

\begin{equation}
	\text{tfidf}(t,d,D) = \text{tf}(t,d)\cdot\lg\frac{N}{1+\left|\left\{d\in D : t\in d\right\}\right|}
	\label{eq:tfidf}
\end{equation}

also where $N$ is the total number of documents in the set of reviews, and
$\left|\left\{d\in D : t\in d\right\}\right|$ is the number of reviews where
the token $t$ occurs. $1$ is added to the denominator to prevent a
division-by-zero scenario, should the token not appear in the review at all.

A conventional \bf{sigmoid function} was used to evaluate the probability
of a review carrying a net positive sentiment, a value from 0 to 1:

\begin{equation}
	p(x) = \frac{1}{1+e^{-{\left(\beta_0+\beta_1x\right)}}}
	\label{eq:sigmoid}
\end{equation}

where $p(x)$ is the probability of the net sentiment of a review being positive, and
$\beta_0+\beta_1x$ is a linear function of \it{tf-idf} $x$.

After the model was constructed, predictions were made using the aforementioned
prediction dataset. Precision-recall and receiver operating characterstic curves
were plotted to evaluate the precision of the model,
and how well the model distinguishes between false positive and true positive predictions.

\section{Results}
\subsection{Research Question 1}
\subsection{Research Question 2}
\subsection{Research Question 3}

\section{Discussion and Future Work}
\subsection{Summary}
\subsection{Limitations}
\subsection{Future Work}
\printbibliography

\section{Appendices}

\subsection{RQ1 Source Code}\label{appendix:srca}
\pycode{../rq1_main.py}

\subsection{RQ2 Source Code}\label{appendix:srcb}
\pycode{../rq2_main.py}

\subsection{RQ3 Source Code: Logistic Regression}\label{appendix:src1}
\pycode{../rq3_logreg.py}

\subsection{RQ3 Source Code: Random Forest Classifier}\label{appendix:src2}
\pycode{../rq3_randomforest.py}

\end{document}
