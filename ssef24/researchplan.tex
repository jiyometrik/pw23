\documentclass[
	fontsize=12pt,
	paper=a4,
	bibliography=totocnumbered, 
]{scrartcl}

\usepackage[top=4cm, left=2.5cm, right=2.5cm, bottom=4cm]{geometry}

\usepackage[
	backend=biber,
	style=apa,
	sorting=nyt
]{biblatex}

\addbibresource{./biblio.bib}

\usepackage{unicode-math}
\protrudechars = 2
\adjustspacing = 2
\newfontfeature{Microtypography}{protrusion = default; expansion = default}
\directlua{fonts.protrusions.setups.default.factor = 0.5}
\setmainfont[Microtypography, Ligatures=TeX]{erewhon}
\setmathfont[Microtypography, Ligatures=TeX]{Erewhon Math}
\setmonofont[Microtypography, Ligatures=TeX, SizeFeatures={Size=10}]{DejaVu Sans Mono}

\usepackage{siunitx}

\usepackage{scrlayer-scrpage}
\ohead{
  \small \normalfont
  \begin{singlespace}
  GROUP\_NO \\
  Sorting sentiments of hotel reviews through machine learning \\
  Yap Hao Ming Darren, Tan Jia He, Fu Jinghang
  \end{singlespace}
}

\usepackage{setspace}
\usepackage{indentfirst}
\setlength\parskip{0.25cm}

\usepackage{xcolor}
\definecolor{links}{RGB}{243, 102, 25} % article title; sections
\usepackage{hyperref}
\urlstyle{tt}
\hypersetup{
	hidelinks,
	colorlinks,
	breaklinks=true,
	urlcolor=links,
	citecolor=links,
	linkcolor=links,
	bookmarksopen=false,
	pdftitle={Research Plan},
	pdfauthor={Yap Hao Ming Darren, Tan Jia He and Fu Jinghang},
}

\addtokomafont{section}{\color{links}\rmfamily}
\addtokomafont{subsection}{\color{links}\rmfamily}

\renewcommand\thesection{\alph{section}}
\renewcommand\thesubsection{\thesection.\roman{subsection}}

\newcommand{\eg}{e.~g.\space}
\newcommand{\ie}{i.~e.\space}

\begin{document}
\pagestyle{scrheadings}
\onehalfspacing

{
	\noindent
	\LARGE\textbf{\underline{Research Plan}}
}

% --- Rationale ---
\section{Rationale}

After the Singapore government relaxed COVID--19 travel restrictions in recent years,
the number of tourists travelling in and out of Singapore has increased.
As such, hotels have seen a rise in the number of tourists to be housed,
and this may lead to an increase in the number of reviews hotels receive.

Today, it is common to use social networks and review websites
to receive data from customer opinions. This is especially true for hotels,
where occupants may evaluate the hotel based on several factors through their
reviews, \eg cleanliness, facilities, location and convenience, etc.
Reviews come in two broad forms: quantitative reviews, based on stars, diamonds, hearts, etc., and
qualitative reviews through short, continuous text.

Quantitative reviews do not always paint the full picture of customers'
opinions towards a certain hotel. Though it is helpful to have a more
objective rating system using numerical scores, \eg the Department of Tourism grading system in
the Philippines, or the European Hotelstars Union system,
these do not reflect the underlying reasons for giving such a rating.
There is also evidence of manipulation of ratings by hotel management itself, where hotels may be
compelled to forge positive or negative ratings to skew the overall rating to serve ulterior motives (\cite{tripadvisor}).
We propose using sentiment analysis to extract customers' opinions on hotels from qualitative reviews as an
alternative approach.

% --- RQ ---
\section{Research Questions and Objectives}

\subsection{Research Questions}
\begin{enumerate}
	\item How could we quantify the sentiments of individual words on a numerical scale?
	\item How could we quantify the sentiments of paragraphs on a numerical scale?
	\item How could we use tokens in hotel reviews to predict the overall sentiment of the review?
\end{enumerate}

\subsection{Objectives}
\begin{enumerate}
	\item To run sentiment analysis on individual words and quantify them on a numerical scale
	\item To run sentiment analysis on paragraphs and quantify them on a numerical scale
	\item To use sentiment analysis on hotel reviews to determine consumers' overall opinions of hotels
\end{enumerate}

% --- Literature ---
\section{Literature Review}

\subsection{Sentiment Analysis}

Sentiment analysis is a field of study which utilises computational methods to analyse text,
and then categorises the text, usually into polarities---positive, neutral
and negative. It has broad applications which range from determining consumers' opinion in
sales and product analysis, to competitor research in marketing, and even detecting public
opinion in social media monitoring. Sentiment analysis will be used in this project to
analyse hotel occupants' reviews, and also determine the most significant upsides
and downsides of each hotel, without interference from human bias.

The stock market opinion of StockTwits communities
of expert investors was predicted. Sentiment analysis was used in a Deep Learning model to extract
sentiment from Big Data. A Pearson Correlation Coefficient combined the linear correlation
between users' sentiment and future stock prices, which
proved the accuracy of user sentiment to be \qty{53}{\percent}.
It was concluded that convolutional neural networks, a type of Deep Learning
artificial neural network, was able to predict stock market movement based on sentiment (\cite{stock}).

Using the social networking site Twitter,
Filipinos' sentiments towards the Philippine government's efforts
at tackling COVID--19 through vaccination programmes were determined (\cite{twitter}).
Sentiment analysis was used to extract sentiment from text in English and multiple Filipino languages,
which was then used to train a Na{\"i}ve--Bayes model. The model classified
sentiments into positive, neutral and negative categories. It was concluded that
the model had a high accuracy of \qty{81.77}{\percent}, even helping the
Philippine government better conduct budget planning and coordinate COVID--19 efforts.

Tourism quality in Spain was analysed (\cite{spain}) by
extracting sentiment from reviews by Chinese tourists on the tourism
social networking sites Baidu Travel, Ctrip, Mafengwo, and Qunar.
Two sentiment analysis methods, lexicon-matching and corpus-based machine
learning methods, were used. These methods allow the processing of
unstructured text of comparatively longer lengths. Clustered data
visualisation categorised aspects of Spanish tourism into positive
and negative groups, with the majority holding a positive sentiment.
It was concluded that sentiment analysis could be used to improve tourism
quality and sustainability decision-making.

SentiStrength, a tool for lexical sentiment analysis was used to study emotions expressed in GitHub
commit comments of different open-source projects (\cite{github}).
Their method involved assigning scores to each word, then calculating the net
score for each comment. SentiStrength splits each comment into snippets, assigns
each a score by computing the maximum and minimum scores of the sentences it contains.
Following which, the average of the positive and negative scores is taken as the
sentiment score of the entire commit. This study showed that Java projects warranted
more negative comments, and projects which had more distributed teams tended
to be received more positively.

\subsection{Conclusion}

In conclusion, the literature reviewed showed many possible applications
of sentiment analysis in quantifying the underlying emotion of feedback
on online platforms. Lexicon-based sentiment analysis, which assigns each
word a sentiment, then calculates a sentence's total sentiment score, can
be used, due to its simplicity in implementation, and the availability of
many open-source sentiment lexicons. In addition, this strategy makes accurate predictions upwards of
\qty{70}{\percent} of the time (\cite{khoo}).

SentiStrength would also be useful for
detecting sentiment from hotel reviews which are usually short in length quickly
and efficiently. Using SentiStrength for sentiment generation is also
rather accurate, generating both positive and negative sentiments with more
than \qty{60}{\percent} accuracy (\cite{thelwall}). Therefore, the strategies listed
above could be adopted or emulated on a smaller scale for this project.

% --- Methodology ---
\section{Methodology}
\subsection{Research Question 1}
\textit{How could we quantify the sentiments of individual words on a numerical scale?}

\begin{enumerate}
	\item Source for a lexicon, as well as a dataset of international hotel reviews in English
	\item Split each review into individual tokens and remove stop words
	\item Assign each token a single sentiment score
\end{enumerate}

\subsection{Research Question 2}
\textit{How could we quantify the sentiments of paragraphs on a numerical scale?}
\begin{enumerate}
	\item Make use of SentiStrength to evaluate positive and negative sentiments
	\item Determine the overall polarity (positive, negative or neutral) of each review
\end{enumerate}

\subsection{Research Question 3}
\textit{How could we use tokens in hotel reviews to predict the overall sentiment of a review?}
\begin{enumerate}
	\item Determine relative token frequency in review dataset
	\item Build logistic regression and random forest classifier machine learning models
	      fed with dataset obtained above
	\item Evaluate accuracy of models by using receiver operating characteristic curve and precision-recall graph
\end{enumerate}

\subsection{Data Sources}

The hotel reviews used in this project were sourced from DataInfiniti's Business Database
(\url{https://datafiniti.co/products/business-data/}) via the open-source database sharing
site Kaggle (\url{https://www.kaggle.com/datasets/datafiniti/hotel-reviews?resource=download}).

% --- Bibliography ---
\printbibliography

\end{document}
